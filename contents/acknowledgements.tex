% !TEX root = ../main.tex

\begin{acknowledgements}
四年光阴倏忽而过,转眼间竟到了本科要毕业的季节。人的认知成长会被各种里程碑事件划分成一个一个的阶段,我往回看总觉得初中三年是一个阶段,高中三年是一个阶段,同一个阶段内的认知水平和世界观似乎没有太大变化。但到了大学之后,认知的成长就走上了快车道。每个学期初从家回到学校的我,和学期末再回到家的我,思想和认知都会有非常大的变化。和四年前的自己对比,似乎我还是我,但许久不见的朋友还是会在几句话后就觉察出我的变化。大学这四年,平心而论,面对新的知识、新的体系,我学习的能力和勇气都有不小的提升。这四年虽说没有多耀眼的成绩,但也安稳度过,记忆中也有点点闪光。

感谢我的父母家人无论在什么时候都是我坚强的支撑,理解和支持我所有的想法和决定,即使是荒诞的想法。

感谢我的同学和朋友,在学习的道路上,互相之间的交流让我免于闭门造车的困扰,尤其是在大三密集的小组作业中,我和队友们共同进退,一起度过了学业最为繁忙的阶段。

感谢我的导师李贤伟老师,在大四这一年的科研生活中,我从零开始学习有关无人机的知识,搭建实验的环境,老师在这过程中给予了我无限的支持,许多琐碎的小事,老师也是亲历亲为。于是在进入这一新领域的不到一年里,虽然缓慢,虽然粗糙,无人机的实验平台已经有了雏形。在论文的撰写过程中,老师耐心地教授我有关写作的细节,小到标题符号,大到文章结构,老师都不厌其烦。感谢四年来为我传道授业解惑的老师,感谢自动化所有的教师,是你们的培养让我不断成长。

衷心感谢百忙之中抽出宝贵时间来评审校阅本论文的老师!
\end{acknowledgements}
