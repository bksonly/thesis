% !TEX root = ../main.tex

\begin{abstract}[zh]
  四旋翼因其高机动性、易于部署和低成本等优势,在农业、救援及娱乐等多个领域得到了广泛应用。尽管如此,其系统的高度非线性以及对环境噪声的敏感性使得四旋翼的精确飞行控制仍然是一大挑战。传统控制策略往往涉及复杂的非线性控制方法或需简化系统模型,这可能导致系统丢失关键动态特性,使得控制效果不佳。
  
  本研究基于旋转矩阵对四旋翼进行无损的动力学建模,在此基础上将欠驱动的四旋翼分解成姿态环路和位置环路,对具有全驱特性的姿态环路应用高阶全驱系统理论,通过对控制量加入非线性补偿项,使姿态环路从非线性系统转化为线性系统,随即对转化出的线性系统应用LQR控制。在理论设计后展开数值仿真,与经典的飞行控制器对比,分析其在多种输入下的收敛速度和跟踪精度,数值实验结果验证了高阶全驱方法在较理想的仿真环境下的优越性。
  
  在数值仿真的基础上,进一步进行了基于ROS的软件在环仿真。在加入了一些工程上的调整后,高阶全驱方法能在更真实的仿真环境中与主流的开源飞控取得相近的控制效果。随后编译生成高阶全驱飞控固件,并烧写进飞控板,进行半实物的硬件在环仿真,验证了飞控固件的有效性。 在仿真后,搭建了室内基于动捕的实机飞行平台,开展了开源飞控下的轨迹跟踪实验并分析了飞行日志。高阶全驱飞控在面对真实世界的非理想因素时还有工程上的欠缺,尚无法取得理想的控制效果。但实机飞行实验积累的真实数据为本方法的进一步实机应用提供了重要支持,本文基于以上所有的实验结果,提出了切实的改进思路。相信在适当的工程改进后,高阶全驱方法有望在现实中取得更好的控制效果。
  
\end{abstract}

\begin{abstract}[en]
  Quadrotors have been widely used in various fields such as agriculture, rescue and recreation due to their advantages of high maneuverability, easy deployment and low cost. Nevertheless, the high nonlinearity of the system and its sensitivity to environmental noise make precise flight control of quadrotors still a major challenge. Traditional control strategies often involve complex nonlinear control methods or require simplified system models, which may lead to the loss of key dynamic characteristics of the system and make the control ineffective.
  
  In this study, we perform a lossless dynamics modeling of the quadrotor based on the rotation matrix, based on which the underdriven quadrotor is decomposed into an attitude loop and a position loop, and apply the theory of higher-order fully-driven system to the attitude loop with fully-driven characteristics, so that the attitude loop is transformed from a nonlinear system to a linear system by adding a nonlinear compensation term to the control quantities, and then the LQR control is applied to the linear system transformed from the attitude loop. Numerical simulations are carried out after the theoretical design to compare with the classical flight controller and analyze its convergence speed and tracking accuracy under multiple inputs, and the numerical experimental results verify the superiority of the higher-order full-drive method in a more ideal simulation environment.
  
  On the basis of numerical simulation, further ROS-based software-in-the-loop simulation is carried out. After adding some engineering adjustments, the higher-order FWD method can achieve similar control results with mainstream open-source flight controls in a more realistic simulation environment. Subsequently, the high-order full-drive flight control firmware was compiled and generated, and burned into the flight control board for semi-physical hardware-in-the-loop simulation to verify the effectiveness of the flight control firmware. After the simulation, an indoor live flight platform based on motion capture was built, and the trajectory tracking experiments under the open-source flight control were carried out and the flight logs were analyzed. The high-order full-drive flight control still has engineering deficiencies when facing real-world non-ideal factors, and is not yet able to achieve ideal control results. However, the real data accumulated from the real flight experiments provide important support for further real-world applications of this method, and this paper proposes practical improvement ideas based on all the above experimental results. It is believed that the higher-order full-drive method is expected to achieve better control effects in reality after appropriate engineering improvements.
\end{abstract}
