% !TEX root = ../main.tex

\begin{abstract}[zh]
  四旋翼飞行器因其高机动性、易于部署和低成本等优势,在农业、救援及娱乐等多个领域得到了广泛应用。尽管如此,其系统的高度非线性以及对环境噪声的敏感性使得四旋翼的精确飞行控制仍然是一大挑战。传统控制策略往往涉及复杂的非线性控制方法或需简化系统模型,这可能导致系统丢失关键动态特性,使得控制效果不佳。
  
  本研究基于旋转矩阵对四旋翼飞行器姿态进行动力学建模,在此基础上将欠驱动的四旋翼飞行器的控制分解成姿态环路和位置环路,对具有全驱特性的姿态环路应用高阶全驱系统理论,通过对控制量加入非线性补偿项,使姿态环路从非线性系统转化为线性系统,进一步对转化出的线性系统应用线性二次型最优控制方法设计最优控制器。在理论设计后展开数值仿真,与经典的飞行控制器对比,分析其在多种输入下的收敛速度和跟踪精度,数值实验结果验证了高阶全驱方法在较理想的仿真环境下的优越性。
  
  在数值仿真的基础上,进一步进行了基于ROS的软件在环仿真。在加入了一些工程上的调整后,高阶全驱方法能在更真实的仿真环境中与主流的开源飞控取得相近的控制效果。随后编译生成高阶全驱飞控固件,进行半实物的硬件在环仿真,验证了所设计的控制器在硬件在环仿真环境下的有效性。 最后,搭建了室内基于动捕的实机飞行平台,开展了开源飞控下的轨迹跟踪实验并分析了飞行日志。初步的实飞实验结果表明,高阶全驱飞控在面对真实世界的非理想因素时还有工程上的欠缺,尚无法取得理想的控制效果。实机飞行实验积累的真实数据为为未来进一步开展本文所提方法的应用基础研究提供了重要参考。
\end{abstract}

\begin{abstract}[en]
  Quadrotors have been widely used in various fields such as agriculture, rescue and recreation due to their advantages of high maneuverability, easy deployment and low cost. Nevertheless, the high nonlinearity of their systems and their sensitivity to environmental noise make precise flight control of quadrotors still a major challenge. Traditional control strategies often involve complex nonlinear control methods or require simplified system models, which may lead to the loss of key dynamic characteristics of the system and make the control ineffective.
  
  In this study, the dynamics of the quadrotor attitude is modeled based on the rotation matrix, on the basis of which the control of the underdriven quadrotor is decomposed into an attitude loop and a position loop, and the higher-order all-drive system theory is applied to the attitude loop with all-drive characteristics, so that the attitude loop is transformed from a nonlinear system to a linear system by adding a nonlinear compensation term to the control quantities, and then the linear system is further transformed by applying the The linear-quadratic optimal control method is applied to the transformed linear system to design the optimal controller. Numerical simulations are carried out after the theoretical design to compare with the classical flight controllers and analyze their convergence speed and tracking accuracy under multiple inputs, and the numerical experimental results verify the superiority of the higher-order full-drive method in a more ideal simulation environment.
  
  On the basis of numerical simulation, further ROS-based software-in-the-loop simulation is carried out. After adding some engineering adjustments, the higher-order FWD method can achieve similar control results with mainstream open-source flight controls in a more realistic simulation environment. Subsequently, the high-order full-drive flight control firmware was compiled and generated for semi-physical hardware-in-the-loop simulation to verify the effectiveness of the designed controller in the hardware-in-the-loop simulation environment. Finally, an indoor live flight platform based on motion capture was built, and the trajectory tracking experiments under the open-source flight control were carried out and the flight logs were analyzed. The preliminary results of the real flight experiments show that the higher-order full-drive flight control still has engineering deficiencies when facing real-world non-ideal factors, and is not yet able to achieve ideal control results. The real data accumulated from the real flight experiments provide an important reference for further applied basic research on the method proposed in this paper.
\end{abstract}
