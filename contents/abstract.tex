% !TEX root = ../main.tex

\begin{abstract}[zh]

  本论文针对四旋翼无人机的运动控制问题,提出了基于高阶全驱系统理论的控制策略。四旋翼无人机因其高度机动性,在农业、救援及娱乐等多个领域得到了广泛应用。尽管如此,其系统的高度非线性以及对环境噪声的敏感性使得精确运动控制成为一大挑战。传统控制策略往往涉及复杂的非线性模型或需简化系统,这可能导致丢失关键动态特性。本研究采用高阶全驱系统理论,直接在非线性模型基础上设计控制器,致力于提升控制精度与响应速度。通过Matlab和基于ROS的仿真实验验证了该控制策略的有效性,并通过搭建实机飞行平台及PX4原生固件的轨迹跟踪实验,为本方法的进一步实机应用提供了重要数据支持。

\end{abstract}

\begin{abstract}[en]
  This thesis addresses the motion control issues of quadrotor UAVs by proposing a control strategy based on the high-order fully actuated (HOFA) system theory. Due to their high mobility, quadrotor UAVs are widely utilized in various fields such as agriculture, rescue operations, and entertainment. Nevertheless, the significant nonlinearity of their systems and sensitivity to environmental noise pose major challenges to achieving precise motion control. Traditional control strategies often involve complex nonlinear models or necessitate simplifying the system, which may result in the loss of critical dynamic features. This study employs the HOFA system theory to design controllers directly based on nonlinear models, aiming to improve control accuracy and response speed. The effectiveness of the proposed control strategy was validated through simulations in Matlab and ROS-based environments. Additionally, real-world flight platforms were constructed, and trajectory tracking experiments using PX4 native firmware were conducted, providing essential data for further real-world applications of this method.
\end{abstract}
