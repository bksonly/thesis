% !TEX root = ../main.tex

\chapter{总结和展望}
\section{全文总结}
在四旋翼的运动控制上,本文提出了姿态环路基于高阶全驱系统理论的最优控制方法。通过Matlab仿真与传统的控制方法对比,验证了本控制方法的优越性。随后展开了基于PX4和ROS的软件在环仿真,进一步验证了本控制策略的有效性。接着进行了硬件在环仿真,搭建了室内基于动捕的无人机实验平台。本文提出的控制方法在接近理想的仿真环境中拥有相对的优越性,但在真实的实验场景下,由于时间的限制,尚有很多工程问题没有解决,还无法得到令人满意的效果。但通过解读PX4原生固件的飞行数据,结合前文的仿真结果,为进一步的研究提供了参考。
\section{非技术性分析}
随着低空经济首次写入政府工作报告,这一新兴经济形态受到了社会各界的广泛关注。低空经济借助无人机等技术的应用,为经济增长注入了新的活力。无人机作为其重要组成部分,其运动控制的精度和稳定性直接影响到其在农业、物流、救援等多个领域的应用效果。

技术进步推动了无人机运动控制系统的不断完善和优化。通过引入先进的控制算法和传感器技术,无人机的操作精度显著提升,使其能够更精准、高效地执行任务。例如,在农业领域,无人机能进行精确的农药喷洒和作物监测,极大提高了生产效率和作物质量。

在物流领域,无人机的应用同样革命性。它们能够实现快速、安全的货物配送,有效降低物流成本并提高配送效率。在紧急救援领域,无人机可以迅速到达灾区,进行救援物资投放和灾情信息收集,为救援工作提供了有力支持。

随着技术的进步和应用场景的不断拓展,低空经济带动了更多新业态和新模式的涌现。这些新兴业态促进了相关产业的发展和升级,形成了更加完善的产业链和生态系统。无人机的运动控制技术的提升,不仅增强了其在现有领域的应用能力,也为低空经济的创新发展开辟了新路径。

综上所述,无人机运动控制的提升对低空经济的贡献巨大。它不仅推动了无人机在多个领域的广泛应用,还促进了低空经济的整体创新与发展,为推动经济增长提供了新的动力。
\section{研究展望}
当前实机HOFA飞行效果不佳,通过前文的分析,的可能原因有以下三点:
\begin{itemize}
  \item 两阶微分求得的期望角加速度会放大噪声的延迟的不良影响
  \item 高频大幅度波动的角速度会被前馈补偿中的二次项放大
  \item 对PX4固件的改写在线程管理上尚有优化余地,控制频率还能进一步提高
\end{itemize}
针对这三点不足,我在此提出相应的解决思路:二阶微分带来的冲击无可避免,即使加强对系统的辨识,优化滤波,二阶微分仍然会引入不小的冲击,那么从工程实际的角度也许只能将其舍弃;二次项造成的噪声放大很难从理论上解决,去掉则高阶全驱将退化成某种误差衡量下的状态反馈,但角速度高频抖动可以通过改良机臂强度和系统辨识后的滤波进行优化;当前PX4固件的改写由于初期的路径,并不是最优的架构,这使得控制周期不是飞控硬件能做到的最优,这可以在后续进行改进。

在以上工程问题之外,在理论方面,等效线性系统可以采取LQR以外的控制方法,也许会更契合系统特性,得到更好的效果。并且,可以进一步研究角速度等估计不准对系统会产生多大程度的影响,以及如何避免此类不良影响。